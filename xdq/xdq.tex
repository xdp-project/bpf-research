%%
%% This is file `sample-xelatex.tex',
%% generated with the docstrip utility.
%%
%% The original source files were:
%%
%% samples.dtx  (with options: `sigconf')
%% 
%% IMPORTANT NOTICE:
%% 
%% For the copyright see the source file.
%% 
%% Any modified versions of this file must be renamed
%% with new filenames distinct from sample-xelatex.tex.
%% 
%% For distribution of the original source see the terms
%% for copying and modification in the file samples.dtx.
%% 
%% This generated file may be distributed as long as the
%% original source files, as listed above, are part of the
%% same distribution. (The sources need not necessarily be
%% in the same archive or directory.)
%%

\documentclass[sigconf, nonacm]{acmart}

\usepackage{mwe}




%%
%% \BibTeX command to typeset BibTeX logo in the docs
\AtBeginDocument{%
  \providecommand\BibTeX{{%
    \normalfont B\kern-0.5em{\scshape i\kern-0.25em b}\kern-0.8em\TeX}}}

\begin{document}

%%
%% The "title" command has an optional parameter,
%% allowing the author to define a "short title" to be used in page headers.
\title{Express Data Queueing}

%%
%% The "author" command and its associated commands are used to define
%% the authors and their affiliations.
%% Of note is the shared affiliation of the first two authors, and the
%% "authornote" and "authornotemark" commands
%% used to denote shared contribution to the research.
\author{Freysteinn Alfreðsson}
\email{freysteinn.alfredsson@kau.se}
\orcid{0000-0003-1516-9370}
\affiliation{%
  \institution{Department of Computer Science}
  \city{Karlstad}
  \country{Sweden}
}
\author{Per Hurtig}
\email{per.hurtig@kau.se}
\affiliation{%
  \institution{Department of Computer Science}
  \city{Karlstad}
  \country{Sweden}
}
%%\orcid{0000-0003-1516-9370}
\author{Anna Brunstrom}
\email{anna.brunstrom@kau.se}
\affiliation{%
  \institution{Department of Computer Science}
  \city{Karlstad}
  \country{Sweden}
}

\author{Toke Høiland-Jørgensen}
\email{toke@redhat.com}
\affiliation{%
  \institution{Red hat}
  \country{Denmark}
}

\author{Jesper Dangaard Brouer}
\email{brouer@redhat.com}
\affiliation{%
  \institution{Red hat}
  \country{Denmark}
}

%%
%% By default, the full list of authors will be used in the page
%% headers. Often, this list is too long, and will overlap
%% other information printed in the page headers. This command allows
%% the author to define a more concise list
%% of authors' names for this purpose.
\renewcommand{\shortauthors}{Alfreðsson, et al.}

\begin{abstract}
\end{abstract}

%%
%% Keywords. The author(s) should pick words that accurately describe
%% the work being presented. Separate the keywords with commas.
\keywords{XDP, eBPF, BPF, Queueing, Linux, Scheduling}

%%
%% This command processes the author and affiliation and title
%% information and builds the first part of the formatted document.
\maketitle

\section{Introduction}

The Linux kernel is a widely used platform in devices that range from IoT
devices, cellphones, home and enterprise routers to servers and cloud offerings.
A common task done by all network equipment and devices is network scheduling,
which the equipment uses to prioritize different traffic between hosts. On
smartphones, the network scheduler might want to prioritize video streaming over
application updates. A cloud provider might wish to schedule fairness between
customers or even prioritize some customers over others. With the growing number
of devices and their diverse network requirements, there is a need for increased
flexibility in the rapid deployment and interaction with schedulers. However,
network equipment and devices tend to have a limited set of network schedulers
available with a few parameters that the customers could control. Therefore,
researchers and network vendors have turned to programmable network schedulers
to allow the customers to write schedulers optimized for their domain-specific
use cases.

Currently, the Linux kernel provides the Queuing Disciplines (Qdiscs) for
handling network scheduling. Adding new Qdiscs requires writing new Linux kernel
modules in the C programming language. However, kernel modules have full access
to the Linux kernel internals and do not provide a safe execution environment. A
minor bug in a Qdisc could crash the whole kernel. Consequently, vendors tend
not to provide third-party Qdiscs because of security reasons. Third-party
providers also need to recompile their modules for every kernel update, which
further limits the deployment of specialized network schedulers. Therefore, to
support programmable schedulers that can be rapidly and safely deployed, we
propose two programmable networking scheduling frameworks for the Linux kernel
using the eBPF execution environment. The eBPF framework is an in-kernel runtime
environment that allows specialized programs to execute within the kernel safely
and without being recompiled. Our first framework is a Qdisc called eBPF-Qdisc,
which allows the programmer to write new schedulers in eBPF that can use all
existing Qdisc infrastructures. The second framework we call Express Data
Queueing (XDQ) and expands upon Express Data Path
(XDP)\cite{hoiland2018express}. XDP is a framework that allows programmers to
attach an eBPF program to the network interface's transmit path. These programs
can alter and redirect the packets, giving the programmer the option of
bypassing the Linux kernel's network stack for added performance, including the
Qdisc layer. XDQ extends on XDP by providing programmers with the ability to
write schedulers for XDP programs.

\begin{itemize}
\item Explain the structure of this technical report.
\end{itemize}

\section{Network Scheduling and Queue Management}

The main function of the network schedulers is to arrange or rearrange packets
in order to meet some specific criteria. However, they can also be used to shape
traffic.

\begin{itemize}
\item Explain the basic concepts of network schedulers.
\item Explain the difference between schedulers and queues.
  \item Give examples of these specific network schedulers:
        \begin{itemize}
          \item DRR
          \item WFQ
        \end{itemize}
\end{itemize}


\section{Network Schedulers in Practice}

\begin{itemize}
  \item Explain what is not possible using today's network schedulers.
  \item Explain why you would like to be able to schedule traffic
\end{itemize}

\subsection{New Requirements}

\begin{itemize}
  \item What are the different systems that demand rapid deployment?
  \item Why can't the default schedulers handle the new requirements?
\end{itemize}


\subsection{eBPF a Game Changer}

\begin{itemize}
  \item Explain why schedulers in eBPF change the field.
  \item Why can't the default schedulers handle the new requirements?
\end{itemize}


\section{Programmable Schedulers}

\begin{itemize}
  \item Explain how programmable schedulers differ from traditional schedulers.
  \item Explain which programmable schedulers exist in practice, such as P4 and others.
  \item Question: What are the names of the current software based programmable schedulers?
        \begin{itemize}
          \item P4
          \item What are the others called? What are the DPDK versions called?
        \end{itemize}
\end{itemize}

% Why:
% - Domain specific requirements


\subsection{Implementation of Programmable Schedulers}

\begin{itemize}
  \item Explain how programmable schedulers are implemented.
  \item Explain that programmable scheduling frameworks provide isolated frameworks to implement schedulers, unlike Qdiscs that are normal kernel modules.
  \item Explain that some frameworks provide special programming languages to describe the schedulers.
  \item Explain that different data structures limit what type of scheduling algorithms we can write using a programmable scheduling framework.
\end{itemize}


\subsubsection{Algorithms}

\begin{itemize}
  \item Explain what types of scheduling algorithms are programmable and which aren't.
  \item Give examples of how they differ.
\end{itemize}


\subsubsection{Data Structures}

\begin{itemize}
  \item Explain how different data-structures affect which scheduling algorithms can be written using programmable schedulers.
  \item Explain the following data-structures and what features they provide:
        \begin{itemize}
          \item PIFO\cite{Sivaraman2016}
          \item SP-PIFO\cite{Alcoz2020}
          \item Eiffel\cite{Saeed2019}
          \item Calendar-queues\cite{KrSharma2020}
        \end{itemize}
\end{itemize}

\section{The extended Berkeley Packet Filter (eBPF)}

The eBPF framework is an in-kernel runtime environment that allows specialized
programs to execute in the kernel safely. These programs attach themselves to
hooks provided by the different subsystems of the kernel. These hooks limit what
the programs can do and define what type of eBPF programs can exist. These types
of programs vary but predominantly, they are related to networking, tracing, and
security. At its core, eBPF is an instruction set. However, as a framework, it
provides a key-value store that enables inter-process communication called eBPF
maps, in-kernel helper functions, and toolchains to create and manage eBPF
programs from user-space.


\subsection{Instruction-set}

The eBPF instruction set is 64-bit and consists of eleven registers listed in
table ~\ref{table:eBPF_registers}. It contains a little over a hundred
instructions. However, the current format defines the opcode as the first octet
of the instruction. Therefore, in the current design, it could only support 256
instructions. The currently available instructions are limited to Arithmetic
Logic Unit (ALU) instructions, byte-swap instructions for Endian handling,
memory, and branch instructions.

\begin{table}
  \caption{Frequency of Special Characters}
  \label{tab:freq}
  \begin{tabular}{ll}
    \toprule
    Register & Type                    \\
    \midrule
    R0       & Return value            \\
    R1-R5    & Parameters to functions \\
    R6-R9    & Callee saved registers  \\
    R10      & Read-only stack pointer \\
    \bottomrule
\end{tabular}
\label{table:eBPF_registers}
\end{table}

The main strength of the eBPF framework comes from the Linux kernel's eBPF
verifier, which ensures that the programs can not crash nor harm the system. It
does this by ensuring that the program does not exceed one million instructions
and that all loops within the program are bounded.

As a second check the verifier checks all conditional logic by pushing the paths
into a stack that it uses to simulate the code until it reaches end of the
program.

\begin{itemize}
  \item Explain how the verifier works.
  \item Explain what type of problems the verifier can protect against.
  \item Explain the difference between the eBPF bytecode and JIT on the architectures that support it.
  \item Explain what eBPF maps are and what different types they provide. Per-cpu core maps, etc.
  \item Question: You mentioned new eBPF maps that were not documented. Which maps are available now that I might not be aware of?
  \item Explain what type of hooks and helper functions the eBPF runtime has.
  \item Explain how eBPF programs are loaded using the bpf system call.
  \item Explain how bpf programs are typically loaded using bcc, libbpf, ip, or tc.
  \item Explain how eBPF is a new technology and tends to have many issues related to the compilers, loaders, and verifier.
\end{itemize}


\section{Express Data Path (XDP)}

XDP is a type of eBPF program that are attached directly to the RX path of a
network interface.

\begin{itemize}
  \item Explain that XDP hooks directly to the RX path of a network driver and allows direct manipulation of the packet.
  \item Explain how XDP allows you to bypass the Linux network stack and redirect or drop the packets.
\end{itemize}


%%\section{Linux Kernel Networking Subsystem}


\section{The Linux Kernel Packet Queueing Discipline (Qdisc)}

\begin{itemize}
  \item Explain where in the network pipe the Qdiscs are located.
  \item Explain the features that Qdiscs provides for traffic control:
        \begin{itemize}
          \item Shaping: Shaping delay packets to meet a desired rate.
          \item Scheduling: Schedulers arrange and/or rearrange packets for output.
          \item Classifying: Classifiers sort or separate traffic into queues.
          \item Policing: Policiers measure and limit traffic in a particular queue.
          \item Dropping: Dropping discards an entire packet, flow or classification.
          \item Marking: Marking is a mechanism by which the packet is altered.
        \end{itemize}
  \item Explain what type of primitives the Qdiscs system supplies:
        \begin{itemize}
          \item Classfull Qdiscs.
          \item Classless Qdiscs.
          \item Actions.
          \item Filters.
        \end{itemize}
  \item The tc command line tool.
\end{itemize}



\subsection{Classless and Classfull Qdiscs}

\begin{itemize}
  \item Explain how the Classfull Qdiscs can create a tree of Qdiscs where the leaves are Classless Qdiscs.
  \item Explain tc command and how it refers to instances of Qdiscs using a 16 bit identifier in hex.
        \item Explain that the tc command parameters may behave differently between Qdics due to each Qdisc handling them differently.
  \item Explain that the ingress Qdiscs are refereed to as the hex number, ffff, and that they can only be classless.
  \item Give an example of using a Qdisc.
\end{itemize}


\subsection{Classifiers and Filters}

\begin{itemize}
  \item Explain that Qdiscs can attach classifiers and filters.
  \item Explain how classifiers are an outer loop that run filters. Also explain that each Qdisc is expected to implement this loop, and that it might not do so.
\end{itemize}


\subsection{eBPF Classifier and Filter}

\begin{itemize}
  \item Explain that the Qdisc layer currently supports two types of eBPF programs. These programs are a eBPF based classifier and filters.
  \item Explain why these eBPF programs
\end{itemize}


\section{Programmable Scheduling Frameworks in eBPF}

\begin{itemize}
  \item Explain what facilities a programmable scheduling framework needs.
        \begin{itemize}
          \item Programmable data structures, where a general data-structure is desired, but more flexible to provide multiple.
          \item A way to dynamically change its functionality on the fly through maps.
        \end{itemize}
  \item Explain which components need to be part of the kernel and can't be in eBPF code due to eBPF limits, such as instruction count.
\end{itemize}


\section{eBPF-Qdisc}

\begin{itemize}
  \item Explain the building blocks of Cong Wang's eBPF-Qdisc.
  \item Explain that is a classfull Qdisc.
  \item Explain what changes are needed to make it better.
  \item Explain the limitations of the eBPF-Qdisc.
\end{itemize}


\section{eXpress Data Queueing (XDQ)}

\begin{itemize}
  \item Explain where XDQ attaches into XDP.
  \item Explain what changes are needed to make it a reality.
\end{itemize}


\section{Conclusion}

\begin{itemize}
  \item Explain the two different solutions, their differences and state.
\end{itemize}



\bibliographystyle{ACM-Reference-Format}
\bibliography{references}


\end{document}
\endinput
